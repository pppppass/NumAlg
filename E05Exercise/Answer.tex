%! TeX encoding = UTF-8
%! TeX program = LuaLaTeX

\documentclass[english, nochinese]{pnote}
\usepackage[paper]{pdef}

\title{Answers to Exercises (Chapter 5)}
\author{Zhihan Li, 1600010653}
\date{December 10, 2018}

\begin{document}

\maketitle

\textbf{Problem 1. (Page 158 Exercise 1)} \textit{Proof.} We have
\begin{equation}
\begin{split}
\phi \rbr{x} + x_{\ast}^{\text{T}} A x_{\ast} &= x^{\text{T}} A x - 2 b^{\text{T}} x + x_{\ast}^{\text{T}} A x_{\ast} \\
&= x^{\text{T}} A x - 2 x_{\ast}^{\text{T}} A x + x_{\ast}^{\text{T}} A x_{\ast} \\
&= \rbr{ x - x_{\ast} }^{\text{T}} A \rbr{ x - x_{\ast} }.
\end{split}
\end{equation}
\hfill$\Box$

\textbf{Problem 2. (Page 158 Exercise 3)} \textit{Proof.} We have
\begin{equation}
x^{\rbr{ k + 1 }} = x^{\rbr{k}} + \alpha \rbr{ r - A x^{\rbr{k}} }
\end{equation}
and
\begin{equation}
b - A x^{\rbr{ k + 1 }} = 0.
\end{equation}
This implies
\begin{equation}
b - A x^{\rbr{k}} = \alpha A \rbr{ b - A x^{\rbr{k}} }.
\end{equation}
\hfill$\Box$

\textbf{Problem 3. (Page 158 Exercise 6)} \textit{Proof.} We have
\begin{equation}
\begin{split}
\phi \rbr{ y_{ i - 1 } + t e_i } &= \rbr{ y_{ i - 1 } + t e_i }^{\text{T}} A \rbr{ y_{ i - 1 } + t e_i } - 2 b^{\text{T}} \rbr{ y_{ i - 1 } + t e_i } \\
&= e_i^{\text{T}} A e_i t^2 + 2 \rbr{ e_i^{\text{T}} A y_{ i - 1 } - 2 e_i^{\text{T}} b } t + \text{const}
\end{split}
\end{equation}
and therefore the $t$ for the $i$-th sub-step satisfies
\begin{equation}
A_{ i i } t + \rbr{ A y_{ i - 1 } }_i = b_i
\end{equation}
or
\begin{equation}
\rbr{ A \rbr{ y_{ i - 1 } + t e_i } }_i = b_i.
\end{equation}
This is exactly the condition we impose for the $i$-th sub-step in Gauss-Seidel iterations, say
\begin{equation}
\rbr{ A y_i }_i = b_i.
\end{equation}
\hfill$\Box$

\textbf{Problem 4. (Page 159 Exercise 9)} \textit{Proof.} We have for $y$
\begin{equation}
\norm{y}_A = \norm{ \sqrt{A} y }_2 \le \norm{\sqrt{A}}_2 \norm{y}_2 = \sqrt{\norm{A}}_2 \norm{y}_2
\end{equation}
and similarly
\begin{equation}
\norm{y}_2 \le \sqrt{\norm{A^{-1}}}_2 \norm{y}_A,
\end{equation}
where
\begin{equation}
\norm{\sqrt{A}}_2 = \sqrt{\norm{A}}_2
\end{equation}
follows from the fact that the eigenvectors of $A$ and $\sqrt{A}$ coincide. Hence,
\begin{equation}
\begin{split}
\norm{ x_k - x_{\ast} }_2 &\le \sqrt{\norm{A^{-1}}}_2 \norm{ x_k - x_{\ast} }_A \\
&\le 2 \sqrt{\norm{A^{-1}}}_2 \rbr{\frac{ \sqrt{\kappa_2} - 1 }{ \sqrt{\kappa_2} + 1 }}^k \norm{ x_0 - x_{\ast} }_A \\
&\le 2 \sqrt{\kappa_2} \rbr{\frac{ \sqrt{\kappa_2} - 1 }{ \sqrt{\kappa_2} + 1 }}^k \norm{ x_0 - x_{\ast} }_2
\end{split}
\end{equation}
as desired.
\hfill$\Box$

\textbf{Problem 5. (Page 159 Exercise 11)} \textit{Proof.} For $ \delta \in \mathcal{X} $ and any $\epsilon$,
\begin{equation}
\begin{split}
&\ptrel{=} \norm{ x_k + \epsilon \delta - A^{-1} b }_A^2 \\
&= \norm{ x_k - A^{-1} b }_A^2 + 2 \epsilon \delta^{\text{T}} A \rbr{ x_k - A^{-1} b } + \epsilon^2 \norm{\delta}_A^2.
\end{split}
\end{equation}
Letting $ \epsilon \rightarrow 0 $,
\begin{equation}
\norm{ x_k + \epsilon \delta - A^{-1} b }_A^2 \ge \nvbr{\norm{ x_k + \epsilon \delta - A^{-1} b }_A^2}_{ \epsilon = 0 } = \norm{ x_k - A^{-1} b }_A^2
\end{equation}
yields
\begin{equation}
\delta^{\text{T}} A \rbr{ x_k - A^{-1} b } = -\delta^{\text{T}} r_k = 0,
\end{equation}
which means $ r_k \perp \mathcal{X} $.
\hfill$\Box$

\end{document}
