%! TeX encoding = UTF-8
%! TeX program = LuaLaTeX

\documentclass[english, nochinese]{pnote}
\usepackage[paper]{pdef}

\title{Answers to Exercises (Week 01)}
\author{Zhihan Li, 1600010653}
\date{September 27, 2018}

\begin{document}

\maketitle

\textbf{Problem 1.} \textit{Proof.} The equation
\begin{equation}
A \rbr{ x_0 + y } = b
\end{equation}
is equivalent to
\begin{equation}
A y = r,
\end{equation}
and therefore it is sufficient to find $ y \in \mathop{\text{span}} \cbr{ r, A r, \cdots, A^{ n - 1 } r } $ satisfying this equation.

Assume $m$ to be the maximal integer such that $ r, A r, \cdots, A^{ m - 1 } r \in \mathbb{R}^n $ are linearly independent. Because $ \dim \mathbb{R}^n = n $, we deduce $ m \le n $. Say
\begin{equation}
a_0 r + a_1 A r + \cdots + a_m A^m r = 0.
\end{equation}Claim
We claim that $ a_0 \neq 0 $. Otherwise, we have
\begin{equation}
A \rbr{ a_1 r + a_2 A r + \cdots + a_m A^{ m - 1 } r } = 0
\end{equation}
and
\begin{equation}
a_1 r + a_2 A r + \cdots + a_m A^{ m - 1 } r = 0
\end{equation}
due to the non-singularity of $A$. However, this contradicts the definition of $m$. As a result, $ a_0 \neq 0 $. Set
\begin{equation}
y = - \rbr{ a_1 r + a_2 A r + \cdots + a_m A^{ m - 1 } r },
\end{equation}
we conclude
\begin{equation}
A y = r
\end{equation}
as desired.
\hfill$\Box$

\textbf{Problem 2.} \textit{Answer.} For $ \epsilon \neq 0 $, the matrix is diagonalizable because
\begin{equation}
\msbr{ 0 & 1 \\ 0 & \epsilon } = \msbr{ 1 & 1 \\ 0 & \epsilon } \msbr{ 0 & \\ & \epsilon } \msbr{ 1 & 1 \\ 0 & \epsilon }^{-1}.
\end{equation}


\end{document}
