%! TeX encoding = UTF-8
%! TeX program = LuaLaTeX

\documentclass[english, nochinese]{pnote}
\usepackage{siunitx}
\usepackage[paper]{pdef}
\usepackage{pgf}

\title{Answers to Exercises (Week 03)}
\author{Zhihan Li, 1600010653}
\date{October 4, 2018}

\begin{document}

\maketitle

\textbf{Problem 1. (Page 38 Exercise 18)} \textit{Proof.} Consider the LU decomposition of $ A = L U $ first. Since $A$ is banded with upper and lower bandwidth $n$, the procedure of LU decomposition yields that $L$ is lower triangular with lower bandwidth $n$. This is because the $i$-th Gauss transformation only involves adding multiplies of $i$-th row to $ i + 1, i + 2, \cdots, i + n $-th rows, and therefore the matrix after $i$-th Gauss transformation has upper and lower bandwidth $n$ as a result. Hence, $L$ has lower bandwidth $n$. Denote the diagonal part of $U$ to be $D$, which turns out to be non-singular since $U$ is non-singular. Thus, we have
\begin{equation}
A = L U = \rbr{ U^{\text{T}} D^{-1} } \rbr{ D L^{\text{T}} }
\end{equation}
and therefore $ L = U^{\text{T}} D^{-1} $, $ A = L D L^{\text{T}} $. Since $A$ is positive definite, we obtain $ D \prec 0 $ and
\begin{equation}
A = \rbr{ L D^{ 1 / 2 } } \rbr{ D^{ 1 / 2 } L^{\text{T}} },
\end{equation}
which means $ L D^{ 1 / 2 } $ is exactly the Cholesky factor and has the lower bandwidth $n$. According to conventions from the textbook, the bandwidth is $ n + 1 $.
\hfill$\Box$

\textbf{Problem 2. (Page 38 Exercise 20)} \textit{Proof.} The existence is exactly part of the argument stated in Problem 1. We have proved the existence of $L$ and $D$ such that
\begin{equation}
A = L D L^{\text{T}}
\end{equation}
regardless of the positive definiteness of $A$. For the uniqueness, suppose another pair, say $ \rbr{ \tilde{L}, \tilde{D} } $, satisfies the equation. We have
\begin{equation}
A = L \rbr{ D L^{\text{T}} } = \tilde{L} \rbr{ \tilde{D} \tilde{L}^{\text{T}}}
\end{equation}
and therefore the uniqueness of LU decomposition yields $ L = \tilde{L} $, $ D L^{\text{T}} = \tilde{D} \tilde{L}^{\text{T}} $, which further implies $ D = \tilde{D} $ since the non-singularity of $L$.
\hfill$\Box$

\textbf{Problem 3. (Page 38 Exercise 23)} \textit{Answer.} We first calculate the LDL\textsuperscript{T} decomposition $ A = L D L^{\text{T}} $, which takes $ \frac{1}{3} n^3 + O \rbr{n^2} $ operations. We then calculate the inverse $L^{-1}$, which can be done by performing Gauss transformation to $L$ and $I$ simultaneously. This takes
\begin{equation}
\sum_{ k = 1 }^n 2 k \rbr{ n - k } = \frac{1}{3} n^3 + O \rbr{n^2}
\end{equation}
operations. We then calculate $ D^{-1} L^{-1} $, which takes $ O \rbr{n^2} $ operators. We finally calculate
\begin{equation}
A^{-1} = L^{-\text{T}} \rbr{ D^{-1} L^{-1} },
\end{equation}
this takes
\begin{equation}
\sum_{ k = 1 }^n \rbr{ n + 1 - k } \rbr{ 2 k - 1 } = \frac{1}{3} n^3 + O \rbr{n^2}
\end{equation}
operations. The final number of operators is $ n^3 + O \rbr{n^2} $, and the temporary space we need is $n^2$.

\textbf{Problem 4.} \textit{Answer.} (1) Denote the numerical solution $u$ as the $N^2$-vector
\begin{equation}
u = \msbr{ u_{ 1 1 } & u_{ 1 2 } & \cdots & u_{ 1 N } & u_{ 2 1 } & u_{ 2 2 } & \cdots & u_{ 2 N } & \cdots & u_{ N 1 } & u_{ 2 2 } & \cdots & u_{ N N } }^{\text{T}}.
\end{equation}
Denote the discretized differential operator $L$ as the $ N^2 \times N^2 $ matrix
\begin{equation}
L = \msbr{ A & -\frac{1}{h^2} I & & & & \\ -\frac{1}{h^2} I & A & -\frac{1}{h^2} I & & & \\ & -\frac{1}{h^2} I & A & \ddots & & \\ & & \ddots & \ddots & \ddots & \\ & & & \ddots & A & -\frac{1}{h^2} I \\ & & & & -\frac{1}{h^2} I & A },
\end{equation}
where $A$ is the $ N \times N $ matrix
\begin{equation}
A = \msbr{ \frac{4}{h^2} & -\frac{1}{h^2} & & & & \\ -\frac{1}{h^2} & \frac{4}{h^2} & -\frac{1}{h^2} & & & \\ & -\frac{1}{h^2} & \frac{4}{h^2} & \ddots & & \\ & & \ddots & \ddots & \ddots & \\ & & & \ddots & \frac{4}{h^2} & -\frac{1}{h^2} \\ & & & & -\frac{1}{h^2} & \frac{4}{h^2} }.
\end{equation}
The discretized equation is
\begin{equation}
L u = 0.
\end{equation}

(2) The running time with respect to $n$ is summarized in Figure \ref{Fig:Time}.

\begin{figure}[htb]
{
\centering
%% Creator: Matplotlib, PGF backend
%%
%% To include the figure in your LaTeX document, write
%%   \input{<filename>.pgf}
%%
%% Make sure the required packages are loaded in your preamble
%%   \usepackage{pgf}
%%
%% Figures using additional raster images can only be included by \input if
%% they are in the same directory as the main LaTeX file. For loading figures
%% from other directories you can use the `import` package
%%   \usepackage{import}
%% and then include the figures with
%%   \import{<path to file>}{<filename>.pgf}
%%
%% Matplotlib used the following preamble
%%   \usepackage{fontspec}
%%   \setmainfont{DejaVuSerif.ttf}[Path=/home/lzh/anaconda3/envs/numalg/lib/python3.7/site-packages/matplotlib/mpl-data/fonts/ttf/]
%%   \setsansfont{DejaVuSans.ttf}[Path=/home/lzh/anaconda3/envs/numalg/lib/python3.7/site-packages/matplotlib/mpl-data/fonts/ttf/]
%%   \setmonofont{DejaVuSansMono.ttf}[Path=/home/lzh/anaconda3/envs/numalg/lib/python3.7/site-packages/matplotlib/mpl-data/fonts/ttf/]
%%
\begingroup%
\makeatletter%
\begin{pgfpicture}%
\pgfpathrectangle{\pgfpointorigin}{\pgfqpoint{6.000000in}{4.000000in}}%
\pgfusepath{use as bounding box, clip}%
\begin{pgfscope}%
\pgfsetbuttcap%
\pgfsetmiterjoin%
\definecolor{currentfill}{rgb}{1.000000,1.000000,1.000000}%
\pgfsetfillcolor{currentfill}%
\pgfsetlinewidth{0.000000pt}%
\definecolor{currentstroke}{rgb}{1.000000,1.000000,1.000000}%
\pgfsetstrokecolor{currentstroke}%
\pgfsetdash{}{0pt}%
\pgfpathmoveto{\pgfqpoint{0.000000in}{0.000000in}}%
\pgfpathlineto{\pgfqpoint{6.000000in}{0.000000in}}%
\pgfpathlineto{\pgfqpoint{6.000000in}{4.000000in}}%
\pgfpathlineto{\pgfqpoint{0.000000in}{4.000000in}}%
\pgfpathclose%
\pgfusepath{fill}%
\end{pgfscope}%
\begin{pgfscope}%
\pgfsetbuttcap%
\pgfsetmiterjoin%
\definecolor{currentfill}{rgb}{1.000000,1.000000,1.000000}%
\pgfsetfillcolor{currentfill}%
\pgfsetlinewidth{0.000000pt}%
\definecolor{currentstroke}{rgb}{0.000000,0.000000,0.000000}%
\pgfsetstrokecolor{currentstroke}%
\pgfsetstrokeopacity{0.000000}%
\pgfsetdash{}{0pt}%
\pgfpathmoveto{\pgfqpoint{0.750000in}{0.500000in}}%
\pgfpathlineto{\pgfqpoint{5.400000in}{0.500000in}}%
\pgfpathlineto{\pgfqpoint{5.400000in}{3.520000in}}%
\pgfpathlineto{\pgfqpoint{0.750000in}{3.520000in}}%
\pgfpathclose%
\pgfusepath{fill}%
\end{pgfscope}%
\begin{pgfscope}%
\pgfpathrectangle{\pgfqpoint{0.750000in}{0.500000in}}{\pgfqpoint{4.650000in}{3.020000in}}%
\pgfusepath{clip}%
\pgfsetbuttcap%
\pgfsetroundjoin%
\definecolor{currentfill}{rgb}{0.121569,0.466667,0.705882}%
\pgfsetfillcolor{currentfill}%
\pgfsetlinewidth{1.003750pt}%
\definecolor{currentstroke}{rgb}{0.121569,0.466667,0.705882}%
\pgfsetstrokecolor{currentstroke}%
\pgfsetdash{}{0pt}%
\pgfsys@defobject{currentmarker}{\pgfqpoint{-0.009821in}{-0.009821in}}{\pgfqpoint{0.009821in}{0.009821in}}{%
\pgfpathmoveto{\pgfqpoint{0.000000in}{-0.009821in}}%
\pgfpathcurveto{\pgfqpoint{0.002605in}{-0.009821in}}{\pgfqpoint{0.005103in}{-0.008786in}}{\pgfqpoint{0.006944in}{-0.006944in}}%
\pgfpathcurveto{\pgfqpoint{0.008786in}{-0.005103in}}{\pgfqpoint{0.009821in}{-0.002605in}}{\pgfqpoint{0.009821in}{0.000000in}}%
\pgfpathcurveto{\pgfqpoint{0.009821in}{0.002605in}}{\pgfqpoint{0.008786in}{0.005103in}}{\pgfqpoint{0.006944in}{0.006944in}}%
\pgfpathcurveto{\pgfqpoint{0.005103in}{0.008786in}}{\pgfqpoint{0.002605in}{0.009821in}}{\pgfqpoint{0.000000in}{0.009821in}}%
\pgfpathcurveto{\pgfqpoint{-0.002605in}{0.009821in}}{\pgfqpoint{-0.005103in}{0.008786in}}{\pgfqpoint{-0.006944in}{0.006944in}}%
\pgfpathcurveto{\pgfqpoint{-0.008786in}{0.005103in}}{\pgfqpoint{-0.009821in}{0.002605in}}{\pgfqpoint{-0.009821in}{0.000000in}}%
\pgfpathcurveto{\pgfqpoint{-0.009821in}{-0.002605in}}{\pgfqpoint{-0.008786in}{-0.005103in}}{\pgfqpoint{-0.006944in}{-0.006944in}}%
\pgfpathcurveto{\pgfqpoint{-0.005103in}{-0.008786in}}{\pgfqpoint{-0.002605in}{-0.009821in}}{\pgfqpoint{0.000000in}{-0.009821in}}%
\pgfpathclose%
\pgfusepath{stroke,fill}%
}%
\begin{pgfscope}%
\pgfsys@transformshift{0.971139in}{0.647024in}%
\pgfsys@useobject{currentmarker}{}%
\end{pgfscope}%
\begin{pgfscope}%
\pgfsys@transformshift{1.013641in}{0.856713in}%
\pgfsys@useobject{currentmarker}{}%
\end{pgfscope}%
\begin{pgfscope}%
\pgfsys@transformshift{1.056144in}{1.695467in}%
\pgfsys@useobject{currentmarker}{}%
\end{pgfscope}%
\begin{pgfscope}%
\pgfsys@transformshift{1.098646in}{1.905156in}%
\pgfsys@useobject{currentmarker}{}%
\end{pgfscope}%
\begin{pgfscope}%
\pgfsys@transformshift{1.141148in}{2.114844in}%
\pgfsys@useobject{currentmarker}{}%
\end{pgfscope}%
\begin{pgfscope}%
\pgfsys@transformshift{1.183650in}{2.114844in}%
\pgfsys@useobject{currentmarker}{}%
\end{pgfscope}%
\begin{pgfscope}%
\pgfsys@transformshift{1.226153in}{2.114844in}%
\pgfsys@useobject{currentmarker}{}%
\end{pgfscope}%
\begin{pgfscope}%
\pgfsys@transformshift{1.268655in}{2.534222in}%
\pgfsys@useobject{currentmarker}{}%
\end{pgfscope}%
\begin{pgfscope}%
\pgfsys@transformshift{1.311157in}{2.324533in}%
\pgfsys@useobject{currentmarker}{}%
\end{pgfscope}%
\begin{pgfscope}%
\pgfsys@transformshift{1.353659in}{2.324533in}%
\pgfsys@useobject{currentmarker}{}%
\end{pgfscope}%
\begin{pgfscope}%
\pgfsys@transformshift{1.396162in}{2.534222in}%
\pgfsys@useobject{currentmarker}{}%
\end{pgfscope}%
\begin{pgfscope}%
\pgfsys@transformshift{1.438664in}{2.324533in}%
\pgfsys@useobject{currentmarker}{}%
\end{pgfscope}%
\begin{pgfscope}%
\pgfsys@transformshift{1.481166in}{2.324533in}%
\pgfsys@useobject{currentmarker}{}%
\end{pgfscope}%
\begin{pgfscope}%
\pgfsys@transformshift{1.523668in}{2.534222in}%
\pgfsys@useobject{currentmarker}{}%
\end{pgfscope}%
\begin{pgfscope}%
\pgfsys@transformshift{1.566170in}{2.324533in}%
\pgfsys@useobject{currentmarker}{}%
\end{pgfscope}%
\begin{pgfscope}%
\pgfsys@transformshift{1.608673in}{2.324533in}%
\pgfsys@useobject{currentmarker}{}%
\end{pgfscope}%
\begin{pgfscope}%
\pgfsys@transformshift{1.651175in}{2.534222in}%
\pgfsys@useobject{currentmarker}{}%
\end{pgfscope}%
\begin{pgfscope}%
\pgfsys@transformshift{1.693677in}{2.743910in}%
\pgfsys@useobject{currentmarker}{}%
\end{pgfscope}%
\begin{pgfscope}%
\pgfsys@transformshift{1.736179in}{2.534222in}%
\pgfsys@useobject{currentmarker}{}%
\end{pgfscope}%
\begin{pgfscope}%
\pgfsys@transformshift{1.778682in}{2.743910in}%
\pgfsys@useobject{currentmarker}{}%
\end{pgfscope}%
\begin{pgfscope}%
\pgfsys@transformshift{1.821184in}{2.743910in}%
\pgfsys@useobject{currentmarker}{}%
\end{pgfscope}%
\begin{pgfscope}%
\pgfsys@transformshift{1.863686in}{2.743910in}%
\pgfsys@useobject{currentmarker}{}%
\end{pgfscope}%
\begin{pgfscope}%
\pgfsys@transformshift{1.906188in}{2.743910in}%
\pgfsys@useobject{currentmarker}{}%
\end{pgfscope}%
\begin{pgfscope}%
\pgfsys@transformshift{1.948691in}{2.743910in}%
\pgfsys@useobject{currentmarker}{}%
\end{pgfscope}%
\begin{pgfscope}%
\pgfsys@transformshift{1.991193in}{2.743910in}%
\pgfsys@useobject{currentmarker}{}%
\end{pgfscope}%
\begin{pgfscope}%
\pgfsys@transformshift{2.033695in}{2.743910in}%
\pgfsys@useobject{currentmarker}{}%
\end{pgfscope}%
\begin{pgfscope}%
\pgfsys@transformshift{2.076197in}{2.743910in}%
\pgfsys@useobject{currentmarker}{}%
\end{pgfscope}%
\begin{pgfscope}%
\pgfsys@transformshift{2.118700in}{2.953599in}%
\pgfsys@useobject{currentmarker}{}%
\end{pgfscope}%
\begin{pgfscope}%
\pgfsys@transformshift{2.161202in}{2.953599in}%
\pgfsys@useobject{currentmarker}{}%
\end{pgfscope}%
\begin{pgfscope}%
\pgfsys@transformshift{2.203704in}{2.953599in}%
\pgfsys@useobject{currentmarker}{}%
\end{pgfscope}%
\begin{pgfscope}%
\pgfsys@transformshift{2.246206in}{2.953599in}%
\pgfsys@useobject{currentmarker}{}%
\end{pgfscope}%
\begin{pgfscope}%
\pgfsys@transformshift{2.288709in}{2.953599in}%
\pgfsys@useobject{currentmarker}{}%
\end{pgfscope}%
\begin{pgfscope}%
\pgfsys@transformshift{2.331211in}{2.953599in}%
\pgfsys@useobject{currentmarker}{}%
\end{pgfscope}%
\begin{pgfscope}%
\pgfsys@transformshift{2.373713in}{3.163287in}%
\pgfsys@useobject{currentmarker}{}%
\end{pgfscope}%
\begin{pgfscope}%
\pgfsys@transformshift{2.416215in}{3.163287in}%
\pgfsys@useobject{currentmarker}{}%
\end{pgfscope}%
\begin{pgfscope}%
\pgfsys@transformshift{2.458718in}{3.163287in}%
\pgfsys@useobject{currentmarker}{}%
\end{pgfscope}%
\begin{pgfscope}%
\pgfsys@transformshift{2.501220in}{3.163287in}%
\pgfsys@useobject{currentmarker}{}%
\end{pgfscope}%
\begin{pgfscope}%
\pgfsys@transformshift{2.543722in}{3.163287in}%
\pgfsys@useobject{currentmarker}{}%
\end{pgfscope}%
\begin{pgfscope}%
\pgfsys@transformshift{2.586224in}{3.163287in}%
\pgfsys@useobject{currentmarker}{}%
\end{pgfscope}%
\begin{pgfscope}%
\pgfsys@transformshift{2.628726in}{3.163287in}%
\pgfsys@useobject{currentmarker}{}%
\end{pgfscope}%
\begin{pgfscope}%
\pgfsys@transformshift{2.671229in}{3.163287in}%
\pgfsys@useobject{currentmarker}{}%
\end{pgfscope}%
\begin{pgfscope}%
\pgfsys@transformshift{2.713731in}{3.163287in}%
\pgfsys@useobject{currentmarker}{}%
\end{pgfscope}%
\begin{pgfscope}%
\pgfsys@transformshift{2.756233in}{3.163287in}%
\pgfsys@useobject{currentmarker}{}%
\end{pgfscope}%
\begin{pgfscope}%
\pgfsys@transformshift{2.798735in}{3.163287in}%
\pgfsys@useobject{currentmarker}{}%
\end{pgfscope}%
\begin{pgfscope}%
\pgfsys@transformshift{2.841238in}{3.163287in}%
\pgfsys@useobject{currentmarker}{}%
\end{pgfscope}%
\begin{pgfscope}%
\pgfsys@transformshift{2.883740in}{3.163287in}%
\pgfsys@useobject{currentmarker}{}%
\end{pgfscope}%
\begin{pgfscope}%
\pgfsys@transformshift{2.926242in}{3.163287in}%
\pgfsys@useobject{currentmarker}{}%
\end{pgfscope}%
\begin{pgfscope}%
\pgfsys@transformshift{2.968744in}{3.163287in}%
\pgfsys@useobject{currentmarker}{}%
\end{pgfscope}%
\begin{pgfscope}%
\pgfsys@transformshift{3.011247in}{3.163287in}%
\pgfsys@useobject{currentmarker}{}%
\end{pgfscope}%
\begin{pgfscope}%
\pgfsys@transformshift{3.053749in}{3.163287in}%
\pgfsys@useobject{currentmarker}{}%
\end{pgfscope}%
\begin{pgfscope}%
\pgfsys@transformshift{3.096251in}{3.163287in}%
\pgfsys@useobject{currentmarker}{}%
\end{pgfscope}%
\begin{pgfscope}%
\pgfsys@transformshift{3.138753in}{3.163287in}%
\pgfsys@useobject{currentmarker}{}%
\end{pgfscope}%
\begin{pgfscope}%
\pgfsys@transformshift{3.181256in}{3.163287in}%
\pgfsys@useobject{currentmarker}{}%
\end{pgfscope}%
\begin{pgfscope}%
\pgfsys@transformshift{3.223758in}{3.163287in}%
\pgfsys@useobject{currentmarker}{}%
\end{pgfscope}%
\begin{pgfscope}%
\pgfsys@transformshift{3.266260in}{3.163287in}%
\pgfsys@useobject{currentmarker}{}%
\end{pgfscope}%
\begin{pgfscope}%
\pgfsys@transformshift{3.308762in}{3.163287in}%
\pgfsys@useobject{currentmarker}{}%
\end{pgfscope}%
\begin{pgfscope}%
\pgfsys@transformshift{3.351265in}{3.163287in}%
\pgfsys@useobject{currentmarker}{}%
\end{pgfscope}%
\begin{pgfscope}%
\pgfsys@transformshift{3.393767in}{3.163287in}%
\pgfsys@useobject{currentmarker}{}%
\end{pgfscope}%
\begin{pgfscope}%
\pgfsys@transformshift{3.436269in}{3.163287in}%
\pgfsys@useobject{currentmarker}{}%
\end{pgfscope}%
\begin{pgfscope}%
\pgfsys@transformshift{3.478771in}{3.163287in}%
\pgfsys@useobject{currentmarker}{}%
\end{pgfscope}%
\begin{pgfscope}%
\pgfsys@transformshift{3.521274in}{3.163287in}%
\pgfsys@useobject{currentmarker}{}%
\end{pgfscope}%
\begin{pgfscope}%
\pgfsys@transformshift{3.563776in}{3.163287in}%
\pgfsys@useobject{currentmarker}{}%
\end{pgfscope}%
\begin{pgfscope}%
\pgfsys@transformshift{3.606278in}{3.163287in}%
\pgfsys@useobject{currentmarker}{}%
\end{pgfscope}%
\begin{pgfscope}%
\pgfsys@transformshift{3.648780in}{3.163287in}%
\pgfsys@useobject{currentmarker}{}%
\end{pgfscope}%
\begin{pgfscope}%
\pgfsys@transformshift{3.691282in}{3.372976in}%
\pgfsys@useobject{currentmarker}{}%
\end{pgfscope}%
\begin{pgfscope}%
\pgfsys@transformshift{3.733785in}{3.163287in}%
\pgfsys@useobject{currentmarker}{}%
\end{pgfscope}%
\begin{pgfscope}%
\pgfsys@transformshift{3.776287in}{3.372976in}%
\pgfsys@useobject{currentmarker}{}%
\end{pgfscope}%
\begin{pgfscope}%
\pgfsys@transformshift{3.818789in}{3.372976in}%
\pgfsys@useobject{currentmarker}{}%
\end{pgfscope}%
\begin{pgfscope}%
\pgfsys@transformshift{3.861291in}{3.163287in}%
\pgfsys@useobject{currentmarker}{}%
\end{pgfscope}%
\begin{pgfscope}%
\pgfsys@transformshift{3.903794in}{3.163287in}%
\pgfsys@useobject{currentmarker}{}%
\end{pgfscope}%
\begin{pgfscope}%
\pgfsys@transformshift{3.946296in}{3.163287in}%
\pgfsys@useobject{currentmarker}{}%
\end{pgfscope}%
\begin{pgfscope}%
\pgfsys@transformshift{3.988798in}{3.163287in}%
\pgfsys@useobject{currentmarker}{}%
\end{pgfscope}%
\begin{pgfscope}%
\pgfsys@transformshift{4.031300in}{3.372976in}%
\pgfsys@useobject{currentmarker}{}%
\end{pgfscope}%
\begin{pgfscope}%
\pgfsys@transformshift{4.073803in}{3.163287in}%
\pgfsys@useobject{currentmarker}{}%
\end{pgfscope}%
\begin{pgfscope}%
\pgfsys@transformshift{4.116305in}{3.372976in}%
\pgfsys@useobject{currentmarker}{}%
\end{pgfscope}%
\begin{pgfscope}%
\pgfsys@transformshift{4.158807in}{3.163287in}%
\pgfsys@useobject{currentmarker}{}%
\end{pgfscope}%
\begin{pgfscope}%
\pgfsys@transformshift{4.201309in}{3.372976in}%
\pgfsys@useobject{currentmarker}{}%
\end{pgfscope}%
\begin{pgfscope}%
\pgfsys@transformshift{4.243812in}{3.163287in}%
\pgfsys@useobject{currentmarker}{}%
\end{pgfscope}%
\begin{pgfscope}%
\pgfsys@transformshift{4.286314in}{3.163287in}%
\pgfsys@useobject{currentmarker}{}%
\end{pgfscope}%
\begin{pgfscope}%
\pgfsys@transformshift{4.328816in}{3.163287in}%
\pgfsys@useobject{currentmarker}{}%
\end{pgfscope}%
\begin{pgfscope}%
\pgfsys@transformshift{4.371318in}{3.163287in}%
\pgfsys@useobject{currentmarker}{}%
\end{pgfscope}%
\begin{pgfscope}%
\pgfsys@transformshift{4.413821in}{3.372976in}%
\pgfsys@useobject{currentmarker}{}%
\end{pgfscope}%
\begin{pgfscope}%
\pgfsys@transformshift{4.456323in}{3.372976in}%
\pgfsys@useobject{currentmarker}{}%
\end{pgfscope}%
\begin{pgfscope}%
\pgfsys@transformshift{4.498825in}{3.372976in}%
\pgfsys@useobject{currentmarker}{}%
\end{pgfscope}%
\begin{pgfscope}%
\pgfsys@transformshift{4.541327in}{3.372976in}%
\pgfsys@useobject{currentmarker}{}%
\end{pgfscope}%
\begin{pgfscope}%
\pgfsys@transformshift{4.583830in}{3.372976in}%
\pgfsys@useobject{currentmarker}{}%
\end{pgfscope}%
\begin{pgfscope}%
\pgfsys@transformshift{4.626332in}{3.372976in}%
\pgfsys@useobject{currentmarker}{}%
\end{pgfscope}%
\begin{pgfscope}%
\pgfsys@transformshift{4.668834in}{3.372976in}%
\pgfsys@useobject{currentmarker}{}%
\end{pgfscope}%
\begin{pgfscope}%
\pgfsys@transformshift{4.711336in}{3.163287in}%
\pgfsys@useobject{currentmarker}{}%
\end{pgfscope}%
\begin{pgfscope}%
\pgfsys@transformshift{4.753838in}{3.163287in}%
\pgfsys@useobject{currentmarker}{}%
\end{pgfscope}%
\begin{pgfscope}%
\pgfsys@transformshift{4.796341in}{3.163287in}%
\pgfsys@useobject{currentmarker}{}%
\end{pgfscope}%
\begin{pgfscope}%
\pgfsys@transformshift{4.838843in}{3.163287in}%
\pgfsys@useobject{currentmarker}{}%
\end{pgfscope}%
\begin{pgfscope}%
\pgfsys@transformshift{4.881345in}{3.163287in}%
\pgfsys@useobject{currentmarker}{}%
\end{pgfscope}%
\begin{pgfscope}%
\pgfsys@transformshift{4.923847in}{3.163287in}%
\pgfsys@useobject{currentmarker}{}%
\end{pgfscope}%
\begin{pgfscope}%
\pgfsys@transformshift{4.966350in}{3.163287in}%
\pgfsys@useobject{currentmarker}{}%
\end{pgfscope}%
\begin{pgfscope}%
\pgfsys@transformshift{5.008852in}{3.372976in}%
\pgfsys@useobject{currentmarker}{}%
\end{pgfscope}%
\begin{pgfscope}%
\pgfsys@transformshift{5.051354in}{3.163287in}%
\pgfsys@useobject{currentmarker}{}%
\end{pgfscope}%
\begin{pgfscope}%
\pgfsys@transformshift{5.093856in}{3.372976in}%
\pgfsys@useobject{currentmarker}{}%
\end{pgfscope}%
\begin{pgfscope}%
\pgfsys@transformshift{5.136359in}{3.372976in}%
\pgfsys@useobject{currentmarker}{}%
\end{pgfscope}%
\begin{pgfscope}%
\pgfsys@transformshift{5.178861in}{3.372976in}%
\pgfsys@useobject{currentmarker}{}%
\end{pgfscope}%
\end{pgfscope}%
\begin{pgfscope}%
\pgfsetbuttcap%
\pgfsetroundjoin%
\definecolor{currentfill}{rgb}{0.000000,0.000000,0.000000}%
\pgfsetfillcolor{currentfill}%
\pgfsetlinewidth{0.803000pt}%
\definecolor{currentstroke}{rgb}{0.000000,0.000000,0.000000}%
\pgfsetstrokecolor{currentstroke}%
\pgfsetdash{}{0pt}%
\pgfsys@defobject{currentmarker}{\pgfqpoint{0.000000in}{-0.048611in}}{\pgfqpoint{0.000000in}{0.000000in}}{%
\pgfpathmoveto{\pgfqpoint{0.000000in}{0.000000in}}%
\pgfpathlineto{\pgfqpoint{0.000000in}{-0.048611in}}%
\pgfusepath{stroke,fill}%
}%
\begin{pgfscope}%
\pgfsys@transformshift{0.928637in}{0.500000in}%
\pgfsys@useobject{currentmarker}{}%
\end{pgfscope}%
\end{pgfscope}%
\begin{pgfscope}%
\definecolor{textcolor}{rgb}{0.000000,0.000000,0.000000}%
\pgfsetstrokecolor{textcolor}%
\pgfsetfillcolor{textcolor}%
\pgftext[x=0.928637in,y=0.402778in,,top]{\color{textcolor}\sffamily\fontsize{10.000000}{12.000000}\selectfont 0}%
\end{pgfscope}%
\begin{pgfscope}%
\pgfsetbuttcap%
\pgfsetroundjoin%
\definecolor{currentfill}{rgb}{0.000000,0.000000,0.000000}%
\pgfsetfillcolor{currentfill}%
\pgfsetlinewidth{0.803000pt}%
\definecolor{currentstroke}{rgb}{0.000000,0.000000,0.000000}%
\pgfsetstrokecolor{currentstroke}%
\pgfsetdash{}{0pt}%
\pgfsys@defobject{currentmarker}{\pgfqpoint{0.000000in}{-0.048611in}}{\pgfqpoint{0.000000in}{0.000000in}}{%
\pgfpathmoveto{\pgfqpoint{0.000000in}{0.000000in}}%
\pgfpathlineto{\pgfqpoint{0.000000in}{-0.048611in}}%
\pgfusepath{stroke,fill}%
}%
\begin{pgfscope}%
\pgfsys@transformshift{1.778682in}{0.500000in}%
\pgfsys@useobject{currentmarker}{}%
\end{pgfscope}%
\end{pgfscope}%
\begin{pgfscope}%
\definecolor{textcolor}{rgb}{0.000000,0.000000,0.000000}%
\pgfsetstrokecolor{textcolor}%
\pgfsetfillcolor{textcolor}%
\pgftext[x=1.778682in,y=0.402778in,,top]{\color{textcolor}\sffamily\fontsize{10.000000}{12.000000}\selectfont 20}%
\end{pgfscope}%
\begin{pgfscope}%
\pgfsetbuttcap%
\pgfsetroundjoin%
\definecolor{currentfill}{rgb}{0.000000,0.000000,0.000000}%
\pgfsetfillcolor{currentfill}%
\pgfsetlinewidth{0.803000pt}%
\definecolor{currentstroke}{rgb}{0.000000,0.000000,0.000000}%
\pgfsetstrokecolor{currentstroke}%
\pgfsetdash{}{0pt}%
\pgfsys@defobject{currentmarker}{\pgfqpoint{0.000000in}{-0.048611in}}{\pgfqpoint{0.000000in}{0.000000in}}{%
\pgfpathmoveto{\pgfqpoint{0.000000in}{0.000000in}}%
\pgfpathlineto{\pgfqpoint{0.000000in}{-0.048611in}}%
\pgfusepath{stroke,fill}%
}%
\begin{pgfscope}%
\pgfsys@transformshift{2.628726in}{0.500000in}%
\pgfsys@useobject{currentmarker}{}%
\end{pgfscope}%
\end{pgfscope}%
\begin{pgfscope}%
\definecolor{textcolor}{rgb}{0.000000,0.000000,0.000000}%
\pgfsetstrokecolor{textcolor}%
\pgfsetfillcolor{textcolor}%
\pgftext[x=2.628726in,y=0.402778in,,top]{\color{textcolor}\sffamily\fontsize{10.000000}{12.000000}\selectfont 40}%
\end{pgfscope}%
\begin{pgfscope}%
\pgfsetbuttcap%
\pgfsetroundjoin%
\definecolor{currentfill}{rgb}{0.000000,0.000000,0.000000}%
\pgfsetfillcolor{currentfill}%
\pgfsetlinewidth{0.803000pt}%
\definecolor{currentstroke}{rgb}{0.000000,0.000000,0.000000}%
\pgfsetstrokecolor{currentstroke}%
\pgfsetdash{}{0pt}%
\pgfsys@defobject{currentmarker}{\pgfqpoint{0.000000in}{-0.048611in}}{\pgfqpoint{0.000000in}{0.000000in}}{%
\pgfpathmoveto{\pgfqpoint{0.000000in}{0.000000in}}%
\pgfpathlineto{\pgfqpoint{0.000000in}{-0.048611in}}%
\pgfusepath{stroke,fill}%
}%
\begin{pgfscope}%
\pgfsys@transformshift{3.478771in}{0.500000in}%
\pgfsys@useobject{currentmarker}{}%
\end{pgfscope}%
\end{pgfscope}%
\begin{pgfscope}%
\definecolor{textcolor}{rgb}{0.000000,0.000000,0.000000}%
\pgfsetstrokecolor{textcolor}%
\pgfsetfillcolor{textcolor}%
\pgftext[x=3.478771in,y=0.402778in,,top]{\color{textcolor}\sffamily\fontsize{10.000000}{12.000000}\selectfont 60}%
\end{pgfscope}%
\begin{pgfscope}%
\pgfsetbuttcap%
\pgfsetroundjoin%
\definecolor{currentfill}{rgb}{0.000000,0.000000,0.000000}%
\pgfsetfillcolor{currentfill}%
\pgfsetlinewidth{0.803000pt}%
\definecolor{currentstroke}{rgb}{0.000000,0.000000,0.000000}%
\pgfsetstrokecolor{currentstroke}%
\pgfsetdash{}{0pt}%
\pgfsys@defobject{currentmarker}{\pgfqpoint{0.000000in}{-0.048611in}}{\pgfqpoint{0.000000in}{0.000000in}}{%
\pgfpathmoveto{\pgfqpoint{0.000000in}{0.000000in}}%
\pgfpathlineto{\pgfqpoint{0.000000in}{-0.048611in}}%
\pgfusepath{stroke,fill}%
}%
\begin{pgfscope}%
\pgfsys@transformshift{4.328816in}{0.500000in}%
\pgfsys@useobject{currentmarker}{}%
\end{pgfscope}%
\end{pgfscope}%
\begin{pgfscope}%
\definecolor{textcolor}{rgb}{0.000000,0.000000,0.000000}%
\pgfsetstrokecolor{textcolor}%
\pgfsetfillcolor{textcolor}%
\pgftext[x=4.328816in,y=0.402778in,,top]{\color{textcolor}\sffamily\fontsize{10.000000}{12.000000}\selectfont 80}%
\end{pgfscope}%
\begin{pgfscope}%
\pgfsetbuttcap%
\pgfsetroundjoin%
\definecolor{currentfill}{rgb}{0.000000,0.000000,0.000000}%
\pgfsetfillcolor{currentfill}%
\pgfsetlinewidth{0.803000pt}%
\definecolor{currentstroke}{rgb}{0.000000,0.000000,0.000000}%
\pgfsetstrokecolor{currentstroke}%
\pgfsetdash{}{0pt}%
\pgfsys@defobject{currentmarker}{\pgfqpoint{0.000000in}{-0.048611in}}{\pgfqpoint{0.000000in}{0.000000in}}{%
\pgfpathmoveto{\pgfqpoint{0.000000in}{0.000000in}}%
\pgfpathlineto{\pgfqpoint{0.000000in}{-0.048611in}}%
\pgfusepath{stroke,fill}%
}%
\begin{pgfscope}%
\pgfsys@transformshift{5.178861in}{0.500000in}%
\pgfsys@useobject{currentmarker}{}%
\end{pgfscope}%
\end{pgfscope}%
\begin{pgfscope}%
\definecolor{textcolor}{rgb}{0.000000,0.000000,0.000000}%
\pgfsetstrokecolor{textcolor}%
\pgfsetfillcolor{textcolor}%
\pgftext[x=5.178861in,y=0.402778in,,top]{\color{textcolor}\sffamily\fontsize{10.000000}{12.000000}\selectfont 100}%
\end{pgfscope}%
\begin{pgfscope}%
\definecolor{textcolor}{rgb}{0.000000,0.000000,0.000000}%
\pgfsetstrokecolor{textcolor}%
\pgfsetfillcolor{textcolor}%
\pgftext[x=3.075000in,y=0.212809in,,top]{\color{textcolor}\sffamily\fontsize{10.000000}{12.000000}\selectfont \(\displaystyle n\)}%
\end{pgfscope}%
\begin{pgfscope}%
\pgfsetbuttcap%
\pgfsetroundjoin%
\definecolor{currentfill}{rgb}{0.000000,0.000000,0.000000}%
\pgfsetfillcolor{currentfill}%
\pgfsetlinewidth{0.803000pt}%
\definecolor{currentstroke}{rgb}{0.000000,0.000000,0.000000}%
\pgfsetstrokecolor{currentstroke}%
\pgfsetdash{}{0pt}%
\pgfsys@defobject{currentmarker}{\pgfqpoint{-0.048611in}{0.000000in}}{\pgfqpoint{0.000000in}{0.000000in}}{%
\pgfpathmoveto{\pgfqpoint{0.000000in}{0.000000in}}%
\pgfpathlineto{\pgfqpoint{-0.048611in}{0.000000in}}%
\pgfusepath{stroke,fill}%
}%
\begin{pgfscope}%
\pgfsys@transformshift{0.750000in}{0.647024in}%
\pgfsys@useobject{currentmarker}{}%
\end{pgfscope}%
\end{pgfscope}%
\begin{pgfscope}%
\definecolor{textcolor}{rgb}{0.000000,0.000000,0.000000}%
\pgfsetstrokecolor{textcolor}%
\pgfsetfillcolor{textcolor}%
\pgftext[x=0.564412in,y=0.594262in,left,base]{\color{textcolor}\sffamily\fontsize{10.000000}{12.000000}\selectfont 4}%
\end{pgfscope}%
\begin{pgfscope}%
\pgfsetbuttcap%
\pgfsetroundjoin%
\definecolor{currentfill}{rgb}{0.000000,0.000000,0.000000}%
\pgfsetfillcolor{currentfill}%
\pgfsetlinewidth{0.803000pt}%
\definecolor{currentstroke}{rgb}{0.000000,0.000000,0.000000}%
\pgfsetstrokecolor{currentstroke}%
\pgfsetdash{}{0pt}%
\pgfsys@defobject{currentmarker}{\pgfqpoint{-0.048611in}{0.000000in}}{\pgfqpoint{0.000000in}{0.000000in}}{%
\pgfpathmoveto{\pgfqpoint{0.000000in}{0.000000in}}%
\pgfpathlineto{\pgfqpoint{-0.048611in}{0.000000in}}%
\pgfusepath{stroke,fill}%
}%
\begin{pgfscope}%
\pgfsys@transformshift{0.750000in}{1.066401in}%
\pgfsys@useobject{currentmarker}{}%
\end{pgfscope}%
\end{pgfscope}%
\begin{pgfscope}%
\definecolor{textcolor}{rgb}{0.000000,0.000000,0.000000}%
\pgfsetstrokecolor{textcolor}%
\pgfsetfillcolor{textcolor}%
\pgftext[x=0.564412in,y=1.013640in,left,base]{\color{textcolor}\sffamily\fontsize{10.000000}{12.000000}\selectfont 6}%
\end{pgfscope}%
\begin{pgfscope}%
\pgfsetbuttcap%
\pgfsetroundjoin%
\definecolor{currentfill}{rgb}{0.000000,0.000000,0.000000}%
\pgfsetfillcolor{currentfill}%
\pgfsetlinewidth{0.803000pt}%
\definecolor{currentstroke}{rgb}{0.000000,0.000000,0.000000}%
\pgfsetstrokecolor{currentstroke}%
\pgfsetdash{}{0pt}%
\pgfsys@defobject{currentmarker}{\pgfqpoint{-0.048611in}{0.000000in}}{\pgfqpoint{0.000000in}{0.000000in}}{%
\pgfpathmoveto{\pgfqpoint{0.000000in}{0.000000in}}%
\pgfpathlineto{\pgfqpoint{-0.048611in}{0.000000in}}%
\pgfusepath{stroke,fill}%
}%
\begin{pgfscope}%
\pgfsys@transformshift{0.750000in}{1.485778in}%
\pgfsys@useobject{currentmarker}{}%
\end{pgfscope}%
\end{pgfscope}%
\begin{pgfscope}%
\definecolor{textcolor}{rgb}{0.000000,0.000000,0.000000}%
\pgfsetstrokecolor{textcolor}%
\pgfsetfillcolor{textcolor}%
\pgftext[x=0.564412in,y=1.433017in,left,base]{\color{textcolor}\sffamily\fontsize{10.000000}{12.000000}\selectfont 8}%
\end{pgfscope}%
\begin{pgfscope}%
\pgfsetbuttcap%
\pgfsetroundjoin%
\definecolor{currentfill}{rgb}{0.000000,0.000000,0.000000}%
\pgfsetfillcolor{currentfill}%
\pgfsetlinewidth{0.803000pt}%
\definecolor{currentstroke}{rgb}{0.000000,0.000000,0.000000}%
\pgfsetstrokecolor{currentstroke}%
\pgfsetdash{}{0pt}%
\pgfsys@defobject{currentmarker}{\pgfqpoint{-0.048611in}{0.000000in}}{\pgfqpoint{0.000000in}{0.000000in}}{%
\pgfpathmoveto{\pgfqpoint{0.000000in}{0.000000in}}%
\pgfpathlineto{\pgfqpoint{-0.048611in}{0.000000in}}%
\pgfusepath{stroke,fill}%
}%
\begin{pgfscope}%
\pgfsys@transformshift{0.750000in}{1.905156in}%
\pgfsys@useobject{currentmarker}{}%
\end{pgfscope}%
\end{pgfscope}%
\begin{pgfscope}%
\definecolor{textcolor}{rgb}{0.000000,0.000000,0.000000}%
\pgfsetstrokecolor{textcolor}%
\pgfsetfillcolor{textcolor}%
\pgftext[x=0.476047in,y=1.852394in,left,base]{\color{textcolor}\sffamily\fontsize{10.000000}{12.000000}\selectfont 10}%
\end{pgfscope}%
\begin{pgfscope}%
\pgfsetbuttcap%
\pgfsetroundjoin%
\definecolor{currentfill}{rgb}{0.000000,0.000000,0.000000}%
\pgfsetfillcolor{currentfill}%
\pgfsetlinewidth{0.803000pt}%
\definecolor{currentstroke}{rgb}{0.000000,0.000000,0.000000}%
\pgfsetstrokecolor{currentstroke}%
\pgfsetdash{}{0pt}%
\pgfsys@defobject{currentmarker}{\pgfqpoint{-0.048611in}{0.000000in}}{\pgfqpoint{0.000000in}{0.000000in}}{%
\pgfpathmoveto{\pgfqpoint{0.000000in}{0.000000in}}%
\pgfpathlineto{\pgfqpoint{-0.048611in}{0.000000in}}%
\pgfusepath{stroke,fill}%
}%
\begin{pgfscope}%
\pgfsys@transformshift{0.750000in}{2.324533in}%
\pgfsys@useobject{currentmarker}{}%
\end{pgfscope}%
\end{pgfscope}%
\begin{pgfscope}%
\definecolor{textcolor}{rgb}{0.000000,0.000000,0.000000}%
\pgfsetstrokecolor{textcolor}%
\pgfsetfillcolor{textcolor}%
\pgftext[x=0.476047in,y=2.271771in,left,base]{\color{textcolor}\sffamily\fontsize{10.000000}{12.000000}\selectfont 12}%
\end{pgfscope}%
\begin{pgfscope}%
\pgfsetbuttcap%
\pgfsetroundjoin%
\definecolor{currentfill}{rgb}{0.000000,0.000000,0.000000}%
\pgfsetfillcolor{currentfill}%
\pgfsetlinewidth{0.803000pt}%
\definecolor{currentstroke}{rgb}{0.000000,0.000000,0.000000}%
\pgfsetstrokecolor{currentstroke}%
\pgfsetdash{}{0pt}%
\pgfsys@defobject{currentmarker}{\pgfqpoint{-0.048611in}{0.000000in}}{\pgfqpoint{0.000000in}{0.000000in}}{%
\pgfpathmoveto{\pgfqpoint{0.000000in}{0.000000in}}%
\pgfpathlineto{\pgfqpoint{-0.048611in}{0.000000in}}%
\pgfusepath{stroke,fill}%
}%
\begin{pgfscope}%
\pgfsys@transformshift{0.750000in}{2.743910in}%
\pgfsys@useobject{currentmarker}{}%
\end{pgfscope}%
\end{pgfscope}%
\begin{pgfscope}%
\definecolor{textcolor}{rgb}{0.000000,0.000000,0.000000}%
\pgfsetstrokecolor{textcolor}%
\pgfsetfillcolor{textcolor}%
\pgftext[x=0.476047in,y=2.691149in,left,base]{\color{textcolor}\sffamily\fontsize{10.000000}{12.000000}\selectfont 14}%
\end{pgfscope}%
\begin{pgfscope}%
\pgfsetbuttcap%
\pgfsetroundjoin%
\definecolor{currentfill}{rgb}{0.000000,0.000000,0.000000}%
\pgfsetfillcolor{currentfill}%
\pgfsetlinewidth{0.803000pt}%
\definecolor{currentstroke}{rgb}{0.000000,0.000000,0.000000}%
\pgfsetstrokecolor{currentstroke}%
\pgfsetdash{}{0pt}%
\pgfsys@defobject{currentmarker}{\pgfqpoint{-0.048611in}{0.000000in}}{\pgfqpoint{0.000000in}{0.000000in}}{%
\pgfpathmoveto{\pgfqpoint{0.000000in}{0.000000in}}%
\pgfpathlineto{\pgfqpoint{-0.048611in}{0.000000in}}%
\pgfusepath{stroke,fill}%
}%
\begin{pgfscope}%
\pgfsys@transformshift{0.750000in}{3.163287in}%
\pgfsys@useobject{currentmarker}{}%
\end{pgfscope}%
\end{pgfscope}%
\begin{pgfscope}%
\definecolor{textcolor}{rgb}{0.000000,0.000000,0.000000}%
\pgfsetstrokecolor{textcolor}%
\pgfsetfillcolor{textcolor}%
\pgftext[x=0.476047in,y=3.110526in,left,base]{\color{textcolor}\sffamily\fontsize{10.000000}{12.000000}\selectfont 16}%
\end{pgfscope}%
\begin{pgfscope}%
\definecolor{textcolor}{rgb}{0.000000,0.000000,0.000000}%
\pgfsetstrokecolor{textcolor}%
\pgfsetfillcolor{textcolor}%
\pgftext[x=0.420492in,y=2.010000in,,bottom,rotate=90.000000]{\color{textcolor}\sffamily\fontsize{10.000000}{12.000000}\selectfont Iterations}%
\end{pgfscope}%
\begin{pgfscope}%
\pgfpathrectangle{\pgfqpoint{0.750000in}{0.500000in}}{\pgfqpoint{4.650000in}{3.020000in}}%
\pgfusepath{clip}%
\pgfsetrectcap%
\pgfsetroundjoin%
\pgfsetlinewidth{1.505625pt}%
\definecolor{currentstroke}{rgb}{0.121569,0.466667,0.705882}%
\pgfsetstrokecolor{currentstroke}%
\pgfsetdash{}{0pt}%
\pgfpathmoveto{\pgfqpoint{0.971139in}{0.647024in}}%
\pgfpathlineto{\pgfqpoint{1.013641in}{0.856713in}}%
\pgfpathlineto{\pgfqpoint{1.056144in}{1.695467in}}%
\pgfpathlineto{\pgfqpoint{1.098646in}{1.905156in}}%
\pgfpathlineto{\pgfqpoint{1.141148in}{2.114844in}}%
\pgfpathlineto{\pgfqpoint{1.183650in}{2.114844in}}%
\pgfpathlineto{\pgfqpoint{1.226153in}{2.114844in}}%
\pgfpathlineto{\pgfqpoint{1.268655in}{2.534222in}}%
\pgfpathlineto{\pgfqpoint{1.311157in}{2.324533in}}%
\pgfpathlineto{\pgfqpoint{1.353659in}{2.324533in}}%
\pgfpathlineto{\pgfqpoint{1.396162in}{2.534222in}}%
\pgfpathlineto{\pgfqpoint{1.438664in}{2.324533in}}%
\pgfpathlineto{\pgfqpoint{1.481166in}{2.324533in}}%
\pgfpathlineto{\pgfqpoint{1.523668in}{2.534222in}}%
\pgfpathlineto{\pgfqpoint{1.566170in}{2.324533in}}%
\pgfpathlineto{\pgfqpoint{1.608673in}{2.324533in}}%
\pgfpathlineto{\pgfqpoint{1.651175in}{2.534222in}}%
\pgfpathlineto{\pgfqpoint{1.693677in}{2.743910in}}%
\pgfpathlineto{\pgfqpoint{1.736179in}{2.534222in}}%
\pgfpathlineto{\pgfqpoint{1.778682in}{2.743910in}}%
\pgfpathlineto{\pgfqpoint{1.821184in}{2.743910in}}%
\pgfpathlineto{\pgfqpoint{1.863686in}{2.743910in}}%
\pgfpathlineto{\pgfqpoint{1.906188in}{2.743910in}}%
\pgfpathlineto{\pgfqpoint{1.948691in}{2.743910in}}%
\pgfpathlineto{\pgfqpoint{1.991193in}{2.743910in}}%
\pgfpathlineto{\pgfqpoint{2.033695in}{2.743910in}}%
\pgfpathlineto{\pgfqpoint{2.076197in}{2.743910in}}%
\pgfpathlineto{\pgfqpoint{2.118700in}{2.953599in}}%
\pgfpathlineto{\pgfqpoint{2.161202in}{2.953599in}}%
\pgfpathlineto{\pgfqpoint{2.203704in}{2.953599in}}%
\pgfpathlineto{\pgfqpoint{2.246206in}{2.953599in}}%
\pgfpathlineto{\pgfqpoint{2.288709in}{2.953599in}}%
\pgfpathlineto{\pgfqpoint{2.331211in}{2.953599in}}%
\pgfpathlineto{\pgfqpoint{2.373713in}{3.163287in}}%
\pgfpathlineto{\pgfqpoint{2.416215in}{3.163287in}}%
\pgfpathlineto{\pgfqpoint{2.458718in}{3.163287in}}%
\pgfpathlineto{\pgfqpoint{2.501220in}{3.163287in}}%
\pgfpathlineto{\pgfqpoint{2.543722in}{3.163287in}}%
\pgfpathlineto{\pgfqpoint{2.586224in}{3.163287in}}%
\pgfpathlineto{\pgfqpoint{2.628726in}{3.163287in}}%
\pgfpathlineto{\pgfqpoint{2.671229in}{3.163287in}}%
\pgfpathlineto{\pgfqpoint{2.713731in}{3.163287in}}%
\pgfpathlineto{\pgfqpoint{2.756233in}{3.163287in}}%
\pgfpathlineto{\pgfqpoint{2.798735in}{3.163287in}}%
\pgfpathlineto{\pgfqpoint{2.841238in}{3.163287in}}%
\pgfpathlineto{\pgfqpoint{2.883740in}{3.163287in}}%
\pgfpathlineto{\pgfqpoint{2.926242in}{3.163287in}}%
\pgfpathlineto{\pgfqpoint{2.968744in}{3.163287in}}%
\pgfpathlineto{\pgfqpoint{3.011247in}{3.163287in}}%
\pgfpathlineto{\pgfqpoint{3.053749in}{3.163287in}}%
\pgfpathlineto{\pgfqpoint{3.096251in}{3.163287in}}%
\pgfpathlineto{\pgfqpoint{3.138753in}{3.163287in}}%
\pgfpathlineto{\pgfqpoint{3.181256in}{3.163287in}}%
\pgfpathlineto{\pgfqpoint{3.223758in}{3.163287in}}%
\pgfpathlineto{\pgfqpoint{3.266260in}{3.163287in}}%
\pgfpathlineto{\pgfqpoint{3.308762in}{3.163287in}}%
\pgfpathlineto{\pgfqpoint{3.351265in}{3.163287in}}%
\pgfpathlineto{\pgfqpoint{3.393767in}{3.163287in}}%
\pgfpathlineto{\pgfqpoint{3.436269in}{3.163287in}}%
\pgfpathlineto{\pgfqpoint{3.478771in}{3.163287in}}%
\pgfpathlineto{\pgfqpoint{3.521274in}{3.163287in}}%
\pgfpathlineto{\pgfqpoint{3.563776in}{3.163287in}}%
\pgfpathlineto{\pgfqpoint{3.606278in}{3.163287in}}%
\pgfpathlineto{\pgfqpoint{3.648780in}{3.163287in}}%
\pgfpathlineto{\pgfqpoint{3.691282in}{3.372976in}}%
\pgfpathlineto{\pgfqpoint{3.733785in}{3.163287in}}%
\pgfpathlineto{\pgfqpoint{3.776287in}{3.372976in}}%
\pgfpathlineto{\pgfqpoint{3.818789in}{3.372976in}}%
\pgfpathlineto{\pgfqpoint{3.861291in}{3.163287in}}%
\pgfpathlineto{\pgfqpoint{3.903794in}{3.163287in}}%
\pgfpathlineto{\pgfqpoint{3.946296in}{3.163287in}}%
\pgfpathlineto{\pgfqpoint{3.988798in}{3.163287in}}%
\pgfpathlineto{\pgfqpoint{4.031300in}{3.372976in}}%
\pgfpathlineto{\pgfqpoint{4.073803in}{3.163287in}}%
\pgfpathlineto{\pgfqpoint{4.116305in}{3.372976in}}%
\pgfpathlineto{\pgfqpoint{4.158807in}{3.163287in}}%
\pgfpathlineto{\pgfqpoint{4.201309in}{3.372976in}}%
\pgfpathlineto{\pgfqpoint{4.243812in}{3.163287in}}%
\pgfpathlineto{\pgfqpoint{4.286314in}{3.163287in}}%
\pgfpathlineto{\pgfqpoint{4.328816in}{3.163287in}}%
\pgfpathlineto{\pgfqpoint{4.371318in}{3.163287in}}%
\pgfpathlineto{\pgfqpoint{4.413821in}{3.372976in}}%
\pgfpathlineto{\pgfqpoint{4.456323in}{3.372976in}}%
\pgfpathlineto{\pgfqpoint{4.498825in}{3.372976in}}%
\pgfpathlineto{\pgfqpoint{4.541327in}{3.372976in}}%
\pgfpathlineto{\pgfqpoint{4.583830in}{3.372976in}}%
\pgfpathlineto{\pgfqpoint{4.626332in}{3.372976in}}%
\pgfpathlineto{\pgfqpoint{4.668834in}{3.372976in}}%
\pgfpathlineto{\pgfqpoint{4.711336in}{3.163287in}}%
\pgfpathlineto{\pgfqpoint{4.753838in}{3.163287in}}%
\pgfpathlineto{\pgfqpoint{4.796341in}{3.163287in}}%
\pgfpathlineto{\pgfqpoint{4.838843in}{3.163287in}}%
\pgfpathlineto{\pgfqpoint{4.881345in}{3.163287in}}%
\pgfpathlineto{\pgfqpoint{4.923847in}{3.163287in}}%
\pgfpathlineto{\pgfqpoint{4.966350in}{3.163287in}}%
\pgfpathlineto{\pgfqpoint{5.008852in}{3.372976in}}%
\pgfpathlineto{\pgfqpoint{5.051354in}{3.163287in}}%
\pgfpathlineto{\pgfqpoint{5.093856in}{3.372976in}}%
\pgfpathlineto{\pgfqpoint{5.136359in}{3.372976in}}%
\pgfpathlineto{\pgfqpoint{5.178861in}{3.372976in}}%
\pgfusepath{stroke}%
\end{pgfscope}%
\begin{pgfscope}%
\pgfsetrectcap%
\pgfsetmiterjoin%
\pgfsetlinewidth{0.803000pt}%
\definecolor{currentstroke}{rgb}{0.000000,0.000000,0.000000}%
\pgfsetstrokecolor{currentstroke}%
\pgfsetdash{}{0pt}%
\pgfpathmoveto{\pgfqpoint{0.750000in}{0.500000in}}%
\pgfpathlineto{\pgfqpoint{0.750000in}{3.520000in}}%
\pgfusepath{stroke}%
\end{pgfscope}%
\begin{pgfscope}%
\pgfsetrectcap%
\pgfsetmiterjoin%
\pgfsetlinewidth{0.803000pt}%
\definecolor{currentstroke}{rgb}{0.000000,0.000000,0.000000}%
\pgfsetstrokecolor{currentstroke}%
\pgfsetdash{}{0pt}%
\pgfpathmoveto{\pgfqpoint{5.400000in}{0.500000in}}%
\pgfpathlineto{\pgfqpoint{5.400000in}{3.520000in}}%
\pgfusepath{stroke}%
\end{pgfscope}%
\begin{pgfscope}%
\pgfsetrectcap%
\pgfsetmiterjoin%
\pgfsetlinewidth{0.803000pt}%
\definecolor{currentstroke}{rgb}{0.000000,0.000000,0.000000}%
\pgfsetstrokecolor{currentstroke}%
\pgfsetdash{}{0pt}%
\pgfpathmoveto{\pgfqpoint{0.750000in}{0.500000in}}%
\pgfpathlineto{\pgfqpoint{5.400000in}{0.500000in}}%
\pgfusepath{stroke}%
\end{pgfscope}%
\begin{pgfscope}%
\pgfsetrectcap%
\pgfsetmiterjoin%
\pgfsetlinewidth{0.803000pt}%
\definecolor{currentstroke}{rgb}{0.000000,0.000000,0.000000}%
\pgfsetstrokecolor{currentstroke}%
\pgfsetdash{}{0pt}%
\pgfpathmoveto{\pgfqpoint{0.750000in}{3.520000in}}%
\pgfpathlineto{\pgfqpoint{5.400000in}{3.520000in}}%
\pgfusepath{stroke}%
\end{pgfscope}%
\end{pgfpicture}%
\makeatother%
\endgroup%

\caption{Running time of different methods}
\label{Fig:Time}
}
{
\small
Here \verb"spsolve" stands for the sparse linear system solver provided by \verb"scipy.sparse.linalg.spsolve", and \verb"solve" stands for the dense linear system solver provided by \verb"numpy.linalg.solve".
}
\end{figure}

It can be seen from the figure that {\color{red}TO BE WRITTEN}.

We also investigate errors of these methods. We compare the $\ell^{\infty}$ error between numerical and analytical solutions in Table \ref{Tbl:NumAna}, and that between numerical solutions in Table \ref{Tbl:NumNum}. In the latter table, the standard solution are chosen to be the ones given by \verb"spsolve".

\begin{table}[htb]
\centering
\small
\begin{tabular}{|c|c|c|c|c|c|c|}
\hline
$n$ & \verb"spsolve" & \verb"solve" & LU & Cholesky & LDL & Banded LU \\
\hline
\input{Table1.tbl}
\end{tabular}
\caption{$\ell^{\infty}$ error between numerical and analytical solution}
\label{Tbl:NumAna}
\end{table}

\begin{table}[htb]
\centering
\small
\begin{tabular}{|c|c|c|c|c|c|c|}
\hline
$n$ & \verb"spsolve" & \verb"solve" & LU & Cholesky & LDL & Banded LU \\
\hline
\input{Table2.tbl}
\end{tabular}
\caption{$\ell^{\infty}$ error between numerical solutions}
\label{Tbl:NumNum}
\end{table}

From these results, we verify the convergence of the numerical discretization schemes as well as the validity of these linear system solvers.

\textbf{Problem 5.} \textit{Answer.} We list the running time, $\ell^{\infty}$ and $\ell^2$ error between yielded and real solutions in Table \ref{Tbl:Time}, \ref{Tbl:ErrInfty} and \ref{Tbl:Err2} respectively. Here $n$ denotes the size of Hilbert matrix and we vary $n$ to explore the behavior of methods when the matrix gets more and more singular.

\begin{table}[htb]
\centering
\begin{tabular}{|c|c|c|c|c|}
\hline
$n$ & \verb"solve" & LU & Cholesky & LDL \\
\hline
\input{Table3.tbl}
\end{tabular}
\caption{Running time (\Si{s}) of different methods}
\label{Tbl:Time}
\end{table}

\begin{table}[htb]
\centering
\begin{tabular}{|c|c|c|c|c|}
\hline
$n$ & \verb"solve" & LU & Cholesky & LDL \\
\hline
\input{Table4.tbl}
\end{tabular}
\caption{$\ell^{\infty}$ error between yielded and real solutions}
\label{Tbl:ErrInfty}
\end{table}

\begin{table}[htb]
\centering
\begin{tabular}{|c|c|c|c|c|}
\hline
$n$ & \verb"solve" & LU & Cholesky & LDL \\
\hline
\input{Table4.tbl}
\end{tabular}
\caption{$\ell^2$ error between yielded and real solutions}
\label{Tbl:Err2}
\end{table}

It can be seen that from these results that the three hand-written methods are slightly slower than packed solver \verb"numpy.linalg.solve". This is because Python is interpreted and hand-written codes takes noticeably much time than compiled ones. Since the Hilbert matrix is highly singular, all the yielded solutions suffer from huge errors, which is caused by accumulation of tiny numerical rounding errors. Experimentally, LDL solver is slightly more stable than LU solver. However, Cholesky decomposition solver fails quickly because the invalidity of square rooting: the diagonal entry happen to be negative at some time because of rounding error and then square rooting directly raises an error, which leads to Not A Number error eventually.

\end{document}
