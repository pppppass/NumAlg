%! TeX encoding = UTF-8
%! TeX program = LuaLaTeX

\documentclass[english, nochinese]{pnote}
\usepackage[paper]{pdef}

\title{Answers to Exercises (Week 08)}
\author{Zhihan Li, 1600010653}
\date{November 12, 2018}

\begin{document}

\maketitle

\textbf{Problem.} \textit{Proof.} Denote $ A_{\epsilon} = D_{\epsilon} - L - L^{\text{T}} $ as in the textbook. Since the iteration scheme is linear, we take $ b = 0 $ without loss of generality. The rules of iterations are
\begin{gather}
x_{ k + 1 }^{\ast} = \rbr{ D_{\epsilon} - L }^{-1} L^{\text{T}} x_k, \\
r_{ k + 1 } = -A_{\epsilon} x_{ k + 1 }^{\ast}, \\
\alpha_{ k + 1 } = \frac{ v^{\text{T}} r_{ k + 1 } }{ v^{\text{T}} A_{\epsilon} v }, \\
x_{ k + 1 } = x_{ k + 1 }^{\ast} + \alpha_{ k + 1 } v,
\end{gather}
where
\begin{equation}
v = \msbr{ 1 & 1 & 1 }^{\text{T}}
\end{equation}
is the eigenvector corresponding to the smallest eigenvalue $\epsilon$. The procedure of deriving $ x_{ k + 1 } $ from $ x_{ k + 1 }^{\ast} $ is actually
\begin{equation}
x_{ k + 1 } = P x_{ k + 1 }^{\ast},
\end{equation}
where $P$ is the orthogonal projection along $v$. The scheme is therefore
\begin{equation}
x_{ k + 1 } = P M_{\epsilon} x_k
\end{equation}
where
\begin{equation}
M_{\epsilon} = \rbr{ D_{\epsilon} - L }^{-1} L^{\text{T}}.
\end{equation}

We notice that
\begin{equation}
M_{\epsilon} = M_0 + O \rbr{\epsilon},
\end{equation}
and moreover $M_0$ has a single eigenvector $v$ for the eigenvalue $1$. This means
\begin{equation}
\rho \rbr{ P M_{\epsilon} } = \rho \rbr{ P M_0 } + O \rbr{\epsilon}.
\end{equation}
We claim
\begin{equation}
\rho \rbr{ P M_0 } < 1.
\end{equation}
This is because the kernel of $ I - M_0 $ is exactly spanned by $v$, which can be verified by direct calculation. For $ x \in \pbr{v} $, we immediately have $ P M_0 v = 0 $. For non-zero $ x \in \mathbb{R}^3 \setminus \pbr{v} $, we have $ \norm{x}_{A_0} > 0 $ and (according to previous exercises)
\begin{equation}
\norm{ P M_0 x }_{A_0}^2 \le \norm{ M_0 x }_{A_0}^2 = \norm{x}_{A_0}^2 - \norm{ x - M_0 x }_{D_0}^2 < \norm{x}_{A_0}^2.
\end{equation}

As a result, $ \rho \rbr{ P M_{\epsilon} } < 1 $ for sufficiently small $\epsilon$ and the convergence rate is independent of $\epsilon$ since $ \rho \rbr{ P M_{\epsilon} } \rightarrow \rho \rbr{ P M_0 } < 1 $.
\hfill$\Box$

\end{document}
